\section{Example 4: Stochastic SIS model}

The models discussed so far have looked at population size: growth, extinction, or simply "wandering about". In the next example we consider a model of infection transmission within a closed popluation (no births or deaths): the susceptible-infected-susceptible (SIS) model.


In this model we consider two types of event:
\begin{itemize}
\item Infection events, which occur in each time step with probability $\beta I dt / N$ per susceptible individual, and
\item Recovery events, which occur in each time step with a probability $\nu dt$ per infected individual.
\end{itemize}


The ODE representation of the model is:

\begin{equation}
\frac{dS}{dt} = - \beta S \frac{I}{N} + \nu I
\frac{dI}{dt} = \beta S \frac{I}{N} - \nu I
\end{equation}


The analytical solution to this ODE can be shown to be:

\begin{equation}
I(t) = I* \frac{1}{ 1 + (I*/I_0 - 1) exp(-(\beta - \nu) t )}
\end{equation}

where $I_0$ is the initial number of infected individuals and 

\begin{equation}
I*=N(1-\nu/\beta) \mbox{.} 
\label{eqn:sis_eqm}
\end{equation}

After sufficient time, the infection and recovery processes balance each other out. This occurs when the number of infectious individuals reaches its equilibrium value I*. We can write the equilibrium value as

\begin{equation}
I* = N (1-1/R_0)
\end{equation}

since $R_0$ is defined as $\beta / \nu$. At this point, we say that the disease is \emph{endemic} in the population.

The stochastic simulation of this system is constructed in a similar way to the previous examples (see the code). Its behaviour, differs significantly from the deterministic simulation for some parameter values. We�ll
investigate these in the rest of this section.

\subsection*{Exercise}

\begin{enumerate}

\item Open the model EXAMPLE4.

\item First, we look at the endemic situation where there is a continuous population of infected individuals
(check in the code that the correct initial condition is in place). 

Using Equation \ref{eqn:sis_eqm} How do you expect the contact rate ($\beta$), recovery rate ($\nu$) and total population ($N$) to affect the equilibrium number of infected individuals?


############################ empty table here#########################

Now try varying these parameters in the model. Does the model behave as you expect?

\item Now look at the accuracy of the deterministic solution. With default parameter values, vary $N$ from around 400 down to about 50. How does deterministic solution compare to the mean value of $I$ from the stochastic simulation?

For large $N$, 

For small $N$,

The differences between the simulations are due to the non-linear infection term, $ \beta SI/N$. The number of susceptibles ($S$) and infected ($I$) are negatively correlated, since when an individual leaves one class, he enters the other. This correlation term is absent from the ODE model which then overestimates the force of infection, giving a higher population of infecteds.

\item Fade-out can occur when the endemic infected population is small.

Set $\beta =0.5$, $\nu=0.3$, and gradually reduce $N$ from the default value of 100. 

From what value do fadeouts start happening? 

Can you explain what�s happening to the difference between mean stochastic behaviour and the deterministic? 

Do a parameter plot of the proportion of simulations that have ``faded out'' by the end of  the simulation, against $N$, from 150 to 10. What do you notice? Estimate a critical community size.

\item Finally, we look at a growing epidemic starting from a single individual. In the code, comment out
the initial condition for endemic behaviour and comment in the epidemic one. Set $N$ high
(300+) and $nsim$ to 1000. What proportion of fade-outs are you getting? 

You should get the same fade-out rate as in example 3. Can you explain why?

\end{enumerate}